\documentclass{article}

\usepackage[margin=0.8in]{geometry}
\usepackage{listings}
\usepackage{enumerate}
\usepackage{amsmath}
\usepackage{color}
\usepackage{graphicx}
\usepackage{tikz}

% \usetikzlibrary{arrows,backgrounds,calc,trees,hobby}

\title{Untitled}
\author{Jonathan Garcia-Mallen}
\date{??? August 2016}
\begin{document}
\lstset{language=Python,
  numbers=left,
  stepnumber=3,    
  firstnumber=1,
  numberfirstline=true
}
\maketitle
\tableofcontents
\section{Background } 
\subsection{This is Duckietown}
\textbf{COPYPASTED FROM DUCKIETOWN.MIT.EDU}\\
""""\\
Duckietown is a class on advanced autonomy taught at MIT
For Spring 2016, MIT has a new class about the science of autonomy at the graduate level. This is a hands-on, project-focused course focusing on self-driving vehicles and high-level autonomy. The problem: **Design the Autonomous Robo-Taxis System for the City of Duckietown.**

    This is a class for makers \&\& thinkers
    This is a collaborative effort: 2 Labs (CSAIL and LIDS), 3 Departments (ME, AeroAstro, EECS), and over a dozen people helped created this class, under the supervision of Prof. John Leonard and Prof. Jon How.
    Duckietown UROP opportunities
    Staff
    Cool videos
\\
Duckietown is a reproducible, open-source class

    This is an open-source class: all materials (hardware design, software, and teaching materials) will be released as "open source" (a free software license for code; a Creative Commons license for teaching materials)
    News and updates
    Follow us on Facebook: facebook.com/duckietown.
    Contacts: Please contact us at duckietown@mit.edu with any question.
""""\\


- this is duckietown (copy from website)
- problem, and why you should care:
    - The students have to ssh into their duckiebots every time they want to 
      test or show off their code.
        - show in excruciating detail what has to be done in order to run a 
        program on the bot, both remote and ssh'ing methods.
    - Problems: it's annoying.  Audience may not understand that the code is
      running entirely on the duckiebot, and not on the laptop. wifi isn't 
      reliable. this has happened: senior faculty pass by quickly, you want
      to show them a demo, but you have to deal with your laptop. This is an
      autonomous vehicle. It shouldn't need a laptop. Thus, we must do
      everything with just a joystick
\section{Requirements and Design Goals}
- Start a program of the student or researcher's choosing without using a
  separate computer to launch it. 
- contained within duckiebot; no laptop interaction.
- reliable. This cannot fail when you're in front of an audience. 
- easy and fast to use. Students shouldn't have a hard time interfacing with
  this, or have to push more buttons than either rosrun method. 
- future-proof. This was not an initial need, but instead grew as a
  requirement as the implementation options' lifespans were reviewed. 
\section{Existing technologies used here}
- js\_linux.py
- init.d/linux
- supervisord
\section{Implementation}
Joystick Daemon
  - apt-get isntall joystick; jstest; python app from github
  - joystick-daemon.py; startuptest
- Starting up the joystick daemon
  - (runlevels, if need more padding)
  - init.d
  - rc.local
  - supervisord
\section{Conclusions and Future work}
- results
- yo. it ran like 10 times in a row, successfully. It be good.
- It shouldn't have taken this long. It's not worth figuring out why
  rc.local didn't work, though. Who knows why.
- Make it prettier. 
- Make it more user friendly. 

\end{document}



